\documentclass{article}

\usepackage{mathtools}

\begin{document}

\centering
\huge MEC 2016\\
\normalsize Rogue QD\\

\bigskip
\bigskip

% buoys 3-100
% boats 150-250
% SOS 500
%

\def \fsos {500\ }
\def \fdl {3\ }
\def \fdh {100\ }
\def \fbl {150\ }
\def \fbh {250\ }

\begin{tabular}{c}
	Ian Prins\\
	Mikhail Andrenkov\\
	Ori Almog\\
\end{tabular}

\newpage
\flushleft

\section{Project Overview} % (fold)

	The RogueQD project aims to supply a simple, clear, and efficient way for naval entities to communicate between one another. By utilizing ITU bands 4 through 6, and covering a frequency domain of 3 - 3,000 KHz, the RogueQD projects sends and receives messages between boats and naval weather buoys. Utilizing both satellite and VLF/LF/MF radio communication, a wide variety of signals can be sent. The RogueQD project also simulates error in message relay between naval entities.

	\bigskip

	Boats navigating the oceans will be able to use the RogueQD project to:

	\begin{itemize}
		\item Get accurate weather data.
		\item Send and respond to SOS distress calls.
		\item Determine its exact location via communication with GPS-equipped buoys. 
	\end{itemize}

	% section project_overview (end)

\section{Assumptions} % (fold)

	\begin{enumerate}
		\item Boats have AM radio transceivers that can transmit and listen along the 3 - 3,000 KHz band.
		\item Weather buoys in the sea are equipped with a transceiver at least as capable as the one the boats are equipped with.
		\item Weather buoys are also equipped with a SCU (Satellite Communications Unit), enabling a GPS unit and centralized relay.
		\item Along the selected frequency domain band, a signal sent should always be received correctly if sent from within 250 km.
		\item Buoys are equipped with weather tracking sensors.
		\item Buoys cannot move.
		\item Boats can move.
	\end{enumerate}

	% section assumptions (end)

\section{Technical Details} % (fold)

	\subsection{Technologies and Infrastructure}

		The RogueQD project is written in Java - a cross-platform and system-compatible programming language. Java is used heavily in computer systems around the world, and is arguably the world's most used programming language. Java can run on Windows, MacOS, Linux, and even *BSD systems!

		\bigskip

		The RogueQD project has taken on an OO (Object Oriented) approach, in order to simplify the architectural structure of the software system. This kind of methodology is particularly useful when modeling real-world entities, as the RogueQD project aims to do.

		\bigskip

		The project used a private repository on GitHub to keep track of the source code. The technology used here is Git - a world-renown system that sees much use in industry.

		\bigskip

		For compilation and testing, the Java 1.8.0\_91 compiler was used, along with a custom makefile for automating compilation process.

		\bigskip

		This document was generated using LaTeX, a universal markup definition for cross-platform document creation. LaTeX utilizes a variety of packages to enable different features, such as indentation, colors, diagrams, and mathematical notation. A list of used packages is supplied in the appendix.

% section technical_details (end)

\section{Internal States} % (fold)

	\subsection{Basic Technical Overview}

		The communication between boats and buoys are the primary function of the system. The frequency band 3 - 3,000 KHz must be shared between all open communication channels.

		\bigskip

		Boats communicate with one another along the \fbl - \fbh KHz range. Buoys communicate with one another along the \fdl - \fdh KHz range. Distress (SOS) signals are propagated along the \fsos KHz frequency.

	\subsection{Definitions and Constants}

		We define the band 3 - 3,000 KHz as $f_{T}$.

		\bigskip

		We define $\Delta$ as a buoy, and $\Delta_{i}$ as the $i^{th}$ buoy in the system. The system can handle $I$ buoys. We define $f_{\Delta_{i}}$ as the frequency domain of the $i^{th}$ buoy.

		\bigskip
		\centering

		$f_{\Delta^{*}} = \bigcup_{i=0}^{I-1}f_{\Delta_{i}}$

		\flushleft
		\bigskip

		Every buoy transmits and receives on its designated frequency, $f_{\Delta_{i}}$. Every buoy also transmits and receives on \fsos KHz, the distress call frequency.

		\bigskip

		Boats can listen and transmit at any frequency (i.e. in $f_{T}$).

		\bigskip

		Some constants are defined below, used in mathematical formulas later on:

		\begin{itemize}
			\item[] $c = 188$
			\item[] $\sigma = 0.3$
		\end{itemize}

	\subsection{Full Technical Overview}

		A buoy $\Delta_{i}$ collects weather data $M_{\omega}$ and sends it on $f_{\Delta_{i}}$. This is a periodic activity. Boats can listen in on this report and acquire $M_{\omega}$.

		\bigskip

		Boats can send a distress signal on \fsos KHz. All buoys that receive the signal send a satellite message $M_{sos}$ to the centralized satellite via their SCU. The satellite then relays $M_{sos}$ to all buoys. At this point all buoys broadcast $M_{sos}$ on $f_{\Delta_{i}}$. This process ensures that anyone listening to a buoy will receive notification of the SOS distress call.

		\bigskip

		A specific band is designated for boat-to-boat communications, that is \fbl to \fbh KHz. On this band there are no buoy interjections. Boats that have planned their journey can agree to speak on a certain frequency without fear of a buoy overlapping with their broadcast.

		\bigskip

		There are two important factors to consider when determining whether or not a signal is received or not:

		\begin{itemize}
			\item Distance
			\item Frequency delta
		\end{itemize}

		The farther away a receiver is, the less likely it is that the message will reach. Also, if the listening frequency differs from the bradcast frequency, chances are slim that the message will be caught.

		\bigskip

		The probability of a message $M$ reaching its recipient can be modeled by:

		\bigskip
		\centering

		$p(M) = \frac{c S_{b} S_{r} e^{(\frac{-(f_{b}-f_{r})^2}{2\sigma^{2}})}}{\sqrt{d2\pi\sigma^{2}}}$

		\flushleft
		\bigskip

		Where $S_{b}$ is the signal broadcast strength, $S_{r}$ is the signal reception strength, $f_{b}$ is the broadcast frequency, $f_{r}$ is the reception frequency, and d is the distance in km.

		\bigskip

		The equation returns a number greater than 0.

		\begin{itemize}
			\item A result that is very close to 0 indicates that the message did not send.
			\item A result that is 1 or larger indicates that the message was sent with complete success.
			\item A result between 0 and 1 indicated that the message was sent, but not entirely successfully. The closer the number is to 1, the better the message quality.
		\end{itemize}

		Messages that sent with a $p(M)$ between 0 and 1 are scrambled and confused. They may contain static, cut out, or otherwise be damaged.

		\bigskip

		The equation represents the mathematical model for normal distribution, with some alterations.



% section internal_states (end)


\end{document}